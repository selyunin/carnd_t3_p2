\chapter{Software Requirements}
\label{ch:requirements}

\section{Lane Departure Warning (LDW) Amplitude Malfunction Software Requirements}

% Instructions: 
% Fill in the software safety requirements for the LDW amplitude malfunction
% technical safety requirements. We have provided the associated technical safety
% requirements. 
 
% Hint: The software safety requirements were discussed in the text from the
% software and hardware lesson.
 
% OPTIONAL:
 
% CHALLENGE ONE
% Develop software safety requirements for the Lane Departure Warning (LDW) frequency
% function and modify the system architecture as needed.
 
% CHALLENGE TWO
% Develop software safety requirements for the Lane Keeping Assistance (LKA) function
% and modify the system architecture as needed.


\begin{table}[!htpb]
%\hspace*{-2.0cm}
\caption{Technical Safety Requirement 01}
\begin{center}
\scriptsize
\renewcommand{\arraystretch}{1.4}
\begin{tabular}{ L{1.8cm}|L{5.2cm}|L{0.7cm}|L{1.3cm}|L{1.8cm}|L{2.0cm}  }
\hline
\rowcolor{black!10}
ID &
Technical Safety Requirement  &
ASIL &
Fault Tolerant Time Interval  &  
Architecture Allocation  &  
Safe State \\\hline
\textcolor{harmonia-blue}{\texttt{Technical Safety Requirement 01}}  &
The LDW safety component shall ensure that the amplitude of the 
  \textcolor{dark-red}{\texttt{LDW\_Torque\_Request}}
  sent to the 
  \textbf{'Final electronic power steering Torque'} component is below 
  \textcolor{dark-red}{\texttt{Max\_Torque\_Amplitude}} &
C &
50 ms  &
  \textbf{LDW Safety} Software Block  &
the lane assistance item is turned off 
\\\hline
\end{tabular}
\end{center}
\label{tab:tr01}
\end{table}



\begin{table}[!htpb]
\caption{Software Safety Requirement}
\begin{center}
\scriptsize
\renewcommand{\arraystretch}{1.4}
\hspace*{-2.0cm}
\begin{tabular}{ L{1.7cm}|L{5.2cm}|L{0.5cm}|L{5.2cm}|L{3.0cm}  }
\hline
\rowcolor{black!10}
ID &
Software Safety Requirement  &
ASIL &
Allocation Software Elements  &  
Safe State \\\hline
\textcolor{harmonia-blue}{\texttt{Software Safety Requirement 01-01}}  &
The input signal 
  \textcolor{dark-red}{\texttt{Primary\_LDW\_Torq\_Req}}
  shall be read and pre-processed to
  determine the torque request coming from the 
  \textbf{Basic/Main LA Functionality} SW
  Component. Signal 
  \textcolor{dark-red}{\texttt{processed\_LDW\_Torq\_Req}}
  shall be generated at the end of the processing
%The LDW safety component shall ensure that the amplitude of the 
%  \textcolor{dark-red}{\texttt{LDW\_Torque\_Request}}
%  sent to the 
%  \textbf{'Final electronic power steering Torque'} component is below 
%  \textcolor{dark-red}{\texttt{Max\_Torque\_Amplitude}} 
  &
C &
  \texttt{LDW\_SAFETY\_INPUT\_PROCESSING}  &
%  \textbf{LDW Safety} Software Block  &
  NA
\\\hline
\textcolor{harmonia-blue}{\texttt{Software Safety Requirement 01-02}}  &
In case the 
  \textcolor{dark-red}{\texttt{processed\_LDW\_Torq\_Req}} 
  signal has a value greater than
  \textcolor{dark-red}{\texttt{Max\_Torque\_Amplitude\_LDW}} 
  (maximum allowed safe torque), the torque signal
  \textcolor{dark-red}{\texttt{limited\_LDW\_Torq\_Req}} 
  shall be set to 0, else 
  \textcolor{dark-red}{\texttt{limited\_LDW\_Torq\_Req}} 
  shall take the value of 
  \textcolor{dark-red}{\texttt{processed\_LDW\_Torq\_Req}}  &
C &
  \texttt{TORQUE\_LIMITER}  &
  \textcolor{dark-red}{\texttt{limited\_LDW\_Torq\_Req}} = 0 (Nm = Newton-meter)
\\\hline
\textcolor{harmonia-blue}{\texttt{Software Safety Requirement 01-03}}  &
  The 
  \textcolor{dark-red}{\texttt{limited\_LDW\_Torq\_Req}}
  shall be transformed into a signal 
  \textcolor{dark-red}{\texttt{LDW\_Torq\_Req}}
  which is suitable to be transmitted outside of the LDW Safety component (
  \textbf{LDW Safety} ) to the 
  \textbf{Final EPS Torque} component. 
  Also see Software Safety Requirement 02-01 and 
  Software Safety Requirement 02-02
  &
C &
\textcolor{dark-red}{\texttt{LDW\_SAFETY\_OUTPUT\_GENERATOR}} &
\textbf{LDW\_Torq\_Req} = 0 (Nm)
\\\hline
\end{tabular}
\end{center}
\label{tab:tr01}
\end{table}




\begin{table}[!htpb]
%\hspace*{-2.0cm}
\caption{Technical Safety Requirement 02}
\begin{center}
\scriptsize
\renewcommand{\arraystretch}{1.4}
\begin{tabular}{ L{1.8cm}|L{5.2cm}|L{0.7cm}|L{1.3cm}|L{1.8cm}|L{2.0cm}  }
\hline
\rowcolor{black!10}
ID &
Technical Safety Requirement  &
ASIL &
Fault Tolerant Time Interval  &  
Architecture Allocation  &  
Safe State \\\hline
\textcolor{harmonia-blue}{\texttt{Technical Safety Requirement 02}}  &
The validity and integrity of the data transmission for 
\textcolor{harmonia-blue}{\texttt{LDW\_Torque\_Request}}
  signal shall be ensured &
C &
50 ms  &
Data Transmission Integrity Check &
N/A
\\\hline
\end{tabular}
\end{center}
\label{tab:tr02}
\end{table}




\begin{table}[!htpb]
\caption{Software Safety Requirement}
\begin{center}
\scriptsize
\renewcommand{\arraystretch}{1.4}
\hspace*{-2.0cm}
\begin{tabular}{ L{1.7cm}|L{5.2cm}|L{0.5cm}|L{5.2cm}|L{3.0cm}  }
\hline
\rowcolor{black!10}
ID &
Software Safety Requirement  &
ASIL &
Allocation Software Elements  &  
Safe State \\\hline
\textcolor{harmonia-blue}{\texttt{Software Safety Requirement 02-01}}  &
Any data to be transmitted outside of the 
LDW Safety 
component (\textbf{LDW Safety})
including 
\textcolor{dark-red}{\texttt{LDW\_Torque\_Req}}
and 
\textcolor{dark-red}{\texttt{activation\_status}}
(see Software Safety Requirement 03-02) shall
be protected by an End2End(E2E) protection mechanism.
&
C &
\texttt{E2ECalc}  &
%  \textbf{LDW Safety} Software Block  &
  LDW\_Torq\_Req = 0 (Nm)
\\\hline
\textcolor{harmonia-blue}{\texttt{Software Safety Requirement 02-02}}  &
The \textbf{E2E} protection protocol shall contain and attach the control data: alive
counter (SQC) and CRC to the data to be transmitted
 &
C &
\texttt{E2ECalc}  &
\textcolor{dark-red}{\texttt{limited\_LDW\_Torq\_Req}} = 0 (Nm = Newton-meter)
\\\hline
\end{tabular}
\end{center}
\label{tab:sr02}
\end{table}



\begin{table}[!htpb]
%\hspace*{-2.0cm}
\caption{Technical Safety Requirement 03}
\begin{center}
\scriptsize
\renewcommand{\arraystretch}{1.4}
\begin{tabular}{ L{1.8cm}|L{5.2cm}|L{0.7cm}|L{1.3cm}|L{1.8cm}|L{2.0cm}  }
\hline
\rowcolor{black!10}
ID &
Technical Safety Requirement  &
ASIL &
Fault Tolerant Time Interval  &  
Architecture Allocation  &  
Safe State \\\hline
\textcolor{harmonia-blue}{\texttt{Technical Safety Requirement 03}}  &
As soon as a failure is detected by the LDW function, it shall
deactivate the \textbf{LDW} feature and the 
\textcolor{dark-red}{\texttt{LDW\_Torque\_Request}}
shall be set to zero
   &
C &
50 ms  &
  \textbf{LDW Safety} Software Block  &
LDW torque output is set to zero
\\\hline
\end{tabular}
\end{center}
\label{tab:tr03}
\end{table}


\begin{table}[!htpb]
\caption{Software Safety Requirement}
\begin{center}
\scriptsize
\renewcommand{\arraystretch}{1.4}
\hspace*{-2.0cm}
\begin{tabular}{ L{1.7cm}|L{5.2cm}|L{0.5cm}|L{5.2cm}|L{3.0cm}  }
\hline
\rowcolor{black!10}
ID &
Software Safety Requirement  &
ASIL &
Allocation Software Elements  &  
Safe State \\\hline
\textcolor{harmonia-blue}{\texttt{Software Safety Requirement 03-01}}  &
Each of the SW elements shall output a signal to indicate any error which is
detected by the element. Error signal =
\textcolor{dark-red}{\texttt{error\_status\_input}} (\textcolor{dark-red}{\texttt{LDW\_SAFETY\_INPUT\_PROCESSING}}),
\textcolor{dark-red}{\texttt{error\_status\_torque\_limiter}} (\textcolor{dark-red}{\texttt{TORQUE\_LIMITER}}),
\textcolor{dark-red}{\texttt{error\_status\_output\_gen}} (\textcolor{dark-red}{\texttt{LDW\_SAFETY\_OUTPUT\_GENERATOR}})
&
C &
\texttt{ALL}  &
N/A
\\\hline
\textcolor{harmonia-blue}{\texttt{Software Safety Requirement 03-02}}  &
A software element shall evaluate the error status of all the other software
elements and in case any 1 of them indicates an error, it shall deactivate
the LDW feature (\textcolor{dark-red}{\texttt{activation\_status}} = 0)
 &
C &
\texttt{LDW\_SAFETY\_ACTIVATION}  &
\texttt{Activation\_status} = 0 (LDW function deactivated)
\\\hline
\textcolor{harmonia-blue}{\texttt{Software Safety Requirement 03-03}}  &
In case of no errors from the software elements, the status of the LDW feature
shall be set to activated (\textcolor{dark-red}{\texttt{activation\_status}} = 1)
&
C &
\texttt{LDW\_SAFETY\_ACTIVATION}  &
N/A
\\\hline
\textcolor{harmonia-blue}{\texttt{Software Safety Requirement 03-04}}  &
In case an error is detected by any of the software elements, it shall set the
value of its corresponding torque to 0 so that 
\textcolor{dark-red}{\texttt{LDW\_Torq\_Req}} is set to 0
&
C &
\texttt{ALL}  &
LDW\_Torq\_Req = 0 (Nm)
\\\hline
\textcolor{harmonia-blue}{\texttt{Software Safety Requirement 03-05}}  &
Once the LDW functionality has been deactivated, it shall stay deactivated till
the time the ignition is switched from off to on again.
&
C &
\texttt{LDW\_SAFETY\_ACTIVATION}  &
\texttt{Activation\_status} = 0 (LDW function deactivated)
\\\hline
\end{tabular}
\end{center}
\label{tab:sr03}
\end{table}



\begin{table}[!htpb]
%\hspace*{-2.0cm}
\caption{Technical Safety Requirement 04}
\begin{center}
\scriptsize
\renewcommand{\arraystretch}{1.4}
\begin{tabular}{ L{1.8cm}|L{5.2cm}|L{0.7cm}|L{1.3cm}|L{1.8cm}|L{2.0cm}  }
\hline
\rowcolor{black!10}
ID &
Technical Safety Requirement  &
ASIL &
Fault Tolerant Time Interval  &  
Architecture Allocation  &  
Safe State \\\hline
\textcolor{harmonia-blue}{\texttt{Technical Safety Requirement 04}}  &
As soon as the LDW function deactivates the LDW feature, the LDW Safety
software block shall send a signal to the car display ECU to turn on a
warning light
C &
50 ms  &
\textbf{LDW Safety} Software Block  &
LDW torque output is set to zero
\\\hline
\end{tabular}
\end{center}
\label{tab:tr04}
\end{table}


\begin{table}[!htpb]
\caption{Software Safety Requirement}
\begin{center}
\scriptsize
\renewcommand{\arraystretch}{1.4}
\hspace*{-2.0cm}
\begin{tabular}{ L{1.7cm}|L{5.2cm}|L{0.5cm}|L{5.2cm}|L{3.0cm}  }
\hline
\rowcolor{black!10}
ID &
Software Safety Requirement  &
ASIL &
Allocation Software Elements  &  
Safe State \\\hline
\textcolor{harmonia-blue}{\texttt{Software Safety Requirement 04-01}}  &
When the LDW function is deactivated 
(\texttt{activation\_status} set to 0), the
\texttt{activation\_status} shall be sent to the car displayECU.
&
C &
\texttt{LDW\_SAFETY\_ACTIVATION}, \texttt{CarDisplay ECU}  &
N/A
\\\hline
\end{tabular}
\end{center}
\label{tab:sr04}
\end{table}





\begin{table}[!htpb]
%\hspace*{-2.0cm}
\caption{Technical Safety Requirement 05}
\begin{center}
\scriptsize
\renewcommand{\arraystretch}{1.4}
\begin{tabular}{ L{1.8cm}|L{5.2cm}|L{0.7cm}|L{1.3cm}|L{1.8cm}|L{2.0cm}  }
\hline
\rowcolor{black!10}
ID &
Technical Safety Requirement  &
ASIL &
Fault Tolerant Time Interval  &  
Architecture Allocation  &  
Safe State \\\hline
\textcolor{harmonia-blue}{\texttt{Technical Safety Requirement 05}}  &
Memory test shall be conducted at start up of the EPS ECU to check for any
  faults in memory
  &
A &
ignition cycle  &
  \textbf{Memory Test} &
LDW torque output is set to zero
\\\hline
\end{tabular}
\end{center}
\label{tab:tr05}
\end{table}


\begin{table}[!htpb]
\caption{Software Safety Requirement}
\begin{center}
\scriptsize
\renewcommand{\arraystretch}{1.4}
\hspace*{-2.0cm}
\begin{tabular}{ L{1.7cm}|L{5.2cm}|L{0.5cm}|L{5.2cm}|L{3.0cm}  }
\hline
\rowcolor{black!10}
ID &
Software Safety Requirement  &
ASIL &
Allocation Software Elements  &  
Safe State \\\hline
\textcolor{harmonia-blue}{\texttt{Software Safety Requirement 05-01}}  &
A CRC verification check over the software code in the Flash memory shall be
done every time the ignition is switched from off to on to check for any
corruption of content.	
&
A &
\texttt{MEMORYTEST}  &
Activation\_status = 0
\\\hline
\textcolor{harmonia-blue}{\texttt{Software Safety Requirement 05-02}}  &
Standard RAM tests to check the data bus, address bus and device integrity
shall be done every time the ignition is switched from off to on (E.g.walking
1s test, RAM pattern test. Refer RAM and processor vendor recommendations )
&
A &
\texttt{MEMORYTEST}  &
Activation\_status = 0
\\\hline
\textcolor{harmonia-blue}{\texttt{Software Safety Requirement 05-03}}  &
The test result of the RAM or Flash memory shall be indicated to the 
\textbf{LDW\_Safety}
component via the 
\textcolor{dark-red}{\texttt{test\_status}} signal	
&
A &
\texttt{MEMORYTEST} &
Activation\_status = 0
\\\hline
\textcolor{harmonia-blue}{\texttt{Software Safety Requirement 05-04}}  &
In case any fault is indicated via the 
\textcolor{dark-red}{\texttt{test\_status}} 
signal the
\textcolor{dark-red}{\texttt{INPUT\_LDW\_PROCESSING}}
shall set an error on 
\textcolor{dark-red}{\texttt{error\_status\_input}} (=1) so that
the LDW functionality is deactivated and the 
\textcolor{dark-red}{\texttt{LDWTorque}}
is set to 0	
&
A &
\texttt{LDW\_SAFETY\_INPUT\_PROCESSING} &
Activation\_status = 0
\\\hline
\end{tabular}
\end{center}
\label{tab:sr05}
\end{table}

