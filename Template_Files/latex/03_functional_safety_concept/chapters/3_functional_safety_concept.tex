\chapter{Functional Safety Concept}
\label{ch:concept}

The functional safety concept consists of:
\begin{itemize}
  \item Functional safety analysis
  \item Functional safety requirements
  \item Functional safety architecture
  \item Warning and degradation concept
\end{itemize}


\section{Functional Safety Analysis}

% Instructions: Fill in the functional safety analysis table below.

\begin{table}[!htpb]
%\hspace*{-2.0cm}
\caption{Violations of the safety goals}
\begin{center}
\scriptsize
\renewcommand{\arraystretch}{1.4}
\begin{tabular}{ L{2.7cm}|L{3.5cm}|L{2.3cm}|L{4.3cm}  }
 \hline
\rowcolor{black!10}
Malfunction ID   &
Main Function of the Item Related to Safety Goal Violations  &
Guidewords (NO, WRONG, EARLY, LATE, MORE, LESS) &
Resulting malfunction  \\\hline
\textcolor{red}{\texttt{Malfunction\_01}}  &
Lane Departure Warning (LDW) function shall apply an oscillating steering
torque to provide the driver a haptic feedback  &
MORE  &
The lane departure warning function applies an oscillating torque with very
high torque frequency (above limit)\\\hline
\textcolor{red}{\texttt{Malfunction\_02}}  &
Lane Departure Warning (LDW) function shall apply an oscillating steering
torque to provide the driver a haptic feedback  &
MORE  &
The lane departure warning function applies an oscillating torque with very
high torque amplitude (above limit)\\\hline
\textcolor{red}{\texttt{Malfunction\_03}}  &
Lane Keeping Assistance (LKA) function shall apply the steering torque when
active in order to stay in ego lane  &
NO  &
The lane keeping assistance function is not limited in time duration which
leads to misuse as an autonomous driving function.\\\hline
%\textcolor{red}{\texttt{Malfunction\_04}}  &
%  &
%  &
% \\\hline
\end{tabular}
\end{center}
\label{tab:malfunctions}
\end{table}


\section{Functional Safety Requirements}

Functional Safety Requirements for the Lane Departure Warning (LDW)
are shown in Table~\ref{tab:FuncRequirementsLdw}.

\begin{table}[!htpb]
%\hspace*{-2.0cm}
\caption{Functional Safety Requirements: Lane Departure Warning}
\begin{center}
\scriptsize
\renewcommand{\arraystretch}{1.4}
\begin{tabular}{ L{1.8cm}|L{3.2cm}|L{0.7cm}|L{3.5cm}|L{3.5cm}  }
 \hline
\rowcolor{black!10}
ID &
Functional Safety Requirement  &
ASIL &
Fault Tolerant Time Interval  &  
Safe State \\\hline
\textcolor{dark-green}{\texttt{Functional Safety Requirement 01-01}}  &
The lane keeping item shall ensure that the lane departure oscillating torque
amplitude is below \textcolor{dark-red}{\texttt{Max\_Torque\_Amplitude}}  &
C &
50 ms  &
the system is turned off
\\\hline
\textcolor{dark-green}{\texttt{Functional Safety Requirement 01-02}}  &
The lane keeping item shall ensure that the lane departure oscillating torque
frequency is below \textcolor{dark-red}{\texttt{Max\_Torque\_Frequency}} &
C &
50 ms  &
the system is turned off 
\\\hline
\end{tabular}
\end{center}
\label{tab:FuncRequirementsLdw}
\end{table}


Lane Departure Warning (LDW) Verification and Validation Acceptance Criteria:
are shown in Table~\ref{tab:LdwVV}.

\begin{table}[!htpb]
%\hspace*{-2.0cm}
\caption{LDW Verification and Validation Acceptance Criteria}
\begin{center}
\scriptsize
\renewcommand{\arraystretch}{1.4}
\begin{tabular}{ L{1.8cm}|L{4.5cm}|L{4.8cm}  }
 \hline
\rowcolor{black!10}
ID &
Validation Acceptance Criteria and Method  &
Verification Acceptance Criteria and Method 
\\\hline
\textcolor{dark-green}{\texttt{Functional Safety Requirement 01-01}}  &
User study: driver's reaction to the \textcolor{dark-red}{\texttt{Max\_Torque\_Amplitude}}
while activating the item &
Fault injection, model checking, formal analysis: meeting the requirement
  upon fault injection\\\hline
\textcolor{dark-green}{\texttt{Functional Safety Requirement 01-02}}  &
User study: driver's reaction to the \textcolor{dark-red}{\texttt{Max\_Torque\_Frequency}}
while activating the item &
Fault injection, model checking, formal analysis: meeting the requirement
  upon injecting fault, performing model checking of the system\\\hline
\end{tabular}
\end{center}
\label{tab:LdwVV}
\end{table}

Functional Safety Requirements for the Lane Keeping Assistance (LKA)
are shown in Table~\ref{tab:FuncRequirementsLka}.

\begin{table}[!htpb]
%\hspace*{-2.0cm}
\caption{Functional Safety Requirements: Lane Keeping Assistance}
\begin{center}
\scriptsize
\renewcommand{\arraystretch}{1.4}
\begin{tabular}{ L{1.8cm}|L{3.2cm}|L{0.7cm}|L{3.5cm}|L{3.5cm}  }
 \hline
\rowcolor{black!10}
ID &
Functional Safety Requirement  &
ASIL &
Fault Tolerant Time Interval  &  
Safe State \\\hline
\textcolor{dark-green}{\texttt{Functional Safety Requirement 02-01}}  &
The electronic power steering ECU shall ensure that the lane keeping
  assistance torque is applied for only \textcolor{dark-red}{\texttt{Max\_Duration}}  &
B &
500 ms  &
The lane keeping functionality is turned off\\\hline
%\textcolor{dark-green}{\texttt{Functional Safety Requirement 01-02}}  &
%  &
%  &
%  &
%\\\hline
\end{tabular}
\end{center}
\label{tab:FuncRequirementsLka}
\end{table}


Lane Keeping Assistance (LKA) Verification and Validation Acceptance Criteria:
are shown in Table~\ref{tab:LkaVV}.

\begin{table}[!htpb]
%\hspace*{-2.0cm}
\caption{LKA Verification and Validation Acceptance Criteria}
\begin{center}
\scriptsize
\renewcommand{\arraystretch}{1.4}
\begin{tabular}{ L{1.8cm}|L{4.5cm}|L{4.8cm}  }
 \hline
\rowcolor{black!10}
ID &
Validation Acceptance Criteria and Method  &
Verification Acceptance Criteria and Method 
\\\hline
\textcolor{dark-green}{\texttt{Functional Safety Requirement 02-01}}  &
User study: \textcolor{dark-red}{\texttt{Max\_Duration}} time interval in
choosen in a way that drivers do not take their hands off the steering wheel  &
Verifying item functionality under fault injections\\\hline
%\textcolor{dark-green}{\texttt{Functional Safety Requirement 02-02}}  &
%  &
%\\\hline
\end{tabular}
\end{center}
\label{tab:LkaVV}
\end{table}


\section{Refinement of the System Architecture}

% Instructions: Include the refined system architecture. 
% Hint: The refined system architecture should include the system architecture
% from the end of the functional safety lesson including all of the ASIL labels.

Figure~\ref{fig:refined-arch} shows the refined architecture including all of the ASIL labels.

\begin{figure}[!htpb]
\includegraphics[width=1.00\linewidth]{graphic_asset_3}
\caption{Refined architecture}
\label{fig:refined-arch}
\end{figure}



\section{Allocation of Functional Safety Requirements to Architecture Elements}

% Instructions: Mark which element or elements are responsible for meeting the
% functional safety requirement. 
% Hint: Only one ECU is responsible for meeting all of the requirements.

Given the refined architecture presented in Figure~\ref{fig:refined-arch}
we allocate each functional safety requirement to a corresponding 
architecture element in Table~\ref{tab:req-arch}.

\begin{table}[!htpb]
%\hspace*{-2.0cm}
\caption{Mapping Functional Safety Requirements to Architecture}
\begin{center}
\scriptsize
\renewcommand{\arraystretch}{1.4}
\begin{tabular}{ L{1.8cm}|L{4.5cm}|L{2.2cm}|L{2.2cm}|L{2.2cm}  }
\hline
\rowcolor{black!10}
ID &
Functional Safety Requirement &
Electronic Power Steering ECU &
Camera ECU &
Car Display ECU
\\\hline
\textcolor{dark-green}{\texttt{Functional Safety Requirement 01-01}}  &
The lane keeping item shall ensure that the lane departure oscillating torque
amplitude is below \textcolor{dark-red}{\texttt{Max\_Torque\_Amplitude}}  &
\checkmark  &
  &
\\\hline
\textcolor{dark-green}{\texttt{Functional Safety Requirement 01-02}}  &
The lane keeping item shall ensure that the lane departure oscillating torque
frequency is below \textcolor{dark-red}{\texttt{Max\_Torque\_Frequency}} &
\checkmark  &
  &
\\\hline
\textcolor{dark-green}{\texttt{Functional Safety Requirement 02-01}}  &
The electronic power steering ECU shall ensure that the lane keeping
assistance torque is applied for only \textcolor{dark-red}{\texttt{Max\_Duration}}  &
\checkmark  &
  &
\\\hline
\end{tabular}
\end{center}
\label{tab:req-arch}
\end{table}


\section{Warning and Degradation Concept}



\begin{table}[!htpb]
%\hspace*{-2.0cm}
\caption{Warning and Degradation Concept}
\begin{center}
\scriptsize
\renewcommand{\arraystretch}{1.4}
\begin{tabular}{ L{1.8cm}|L{2.5cm}|L{4.2cm}|L{2.2cm}|L{2.2cm}  }
\hline
\rowcolor{black!10}
ID &
Degradation Mode &
Trigger for Degradation Mode &
Safe State invoked? &
Driver Warning
\\\hline
\textcolor{brown}{\texttt{WDC-01}}  &
Turn off the functionality &
Vibration Torque frequency is above \textcolor{dark-red}{\texttt{Max\_Torque\_Frequency}}
and vibration torque amplitude is above \textcolor{dark-red}{\texttt{Max\_Torque\_Amplitude}} &
Yes  &
Indication light on car display on\\\hline
\textcolor{brown}{\texttt{WDC-02}}  &
Turn off the functionality  &
Lane keeing assistance is applied for more then \textcolor{dark-red}{\texttt{Max\_Duration}} 
time interval &
Yes  &
Indication light on car display on\\\hline
\end{tabular}
\end{center}
\label{tab:wdc}
\end{table}
