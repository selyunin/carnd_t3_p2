\chapter{Technical Safety Concept}
\label{ch:concept}

\section{Technical Safety Requirements}

% Instructions: Fill in the technical safety requirements for the lane departure
% warning first functional safety requirement. We have provided the associated
% functional safety requirement in the first table below. 
% 
% Hint: The technical safety requirements were discussed in the lesson videos.
% The architecture allocation column should contain element names such as LDW
% Safety block, Data Transmission Integrity Check, etc.  Allocating the technical
% safety requirements to the "EPS ECU" does not provide enough detail for a
% technical safety concept.

\subsection{Lane Departure Warning (LDW) Requirements}

Functional Safety Requirement 01-01 with its associated system elements
(derived in the functional safety concept) presented in
Table~\ref{tab:func0101}.

\begin{table}[!htpb]
%\hspace*{-2.0cm}
\caption{Functional Safety Requirement}
\begin{center}
\scriptsize
\renewcommand{\arraystretch}{1.4}
\begin{tabular}{ L{1.8cm}|L{4.5cm}|L{2.2cm}|L{2.2cm}|L{2.2cm}  }
\hline
\rowcolor{black!10}
ID &
Functional Safety Requirement &
Electronic Power Steering ECU &
Camera ECU &
Car Display ECU
\\\hline
\textcolor{dark-green}{\texttt{Functional Safety Requirement 01-01}}  &
The lane keeping item shall ensure that the lane departure oscillating torque
amplitude is below \textcolor{dark-red}{\texttt{Max\_Torque\_Amplitude}}  &
\checkmark  &
  &
\\\hline
\end{tabular}
\end{center}
\label{tab:func0101}
\end{table}

Technical Safety Requirements related to Functional Safety Requirement 01-01 
presented in the Table~\ref{tab:tr0101}


\begin{table}[!htpb]
%\hspace*{-2.0cm}
\caption{Technical Safety Requirements for \textcolor{dark-green}{\texttt{Functional Safety Requirement 01-01}}}
\begin{center}
\scriptsize
\renewcommand{\arraystretch}{1.4}
\begin{tabular}{ L{1.8cm}|L{5.2cm}|L{0.7cm}|L{1.3cm}|L{1.8cm}|L{2.0cm}  }
\hline
\rowcolor{black!10}
ID &
Technical Safety Requirement  &
ASIL &
Fault Tolerant Time Interval  &  
Architecture Allocation  &  
Safe State \\\hline
\textcolor{harmonia-blue}{\texttt{Technical Safety Requirement 01-01-01}}  &
The LDW safety component shall ensure that the amplitude of the 
  \textcolor{dark-red}{\texttt{LDW\_Torque\_Request}}
  sent to the 
  \textbf{'Final electronic power steering Torque'} component is below 
  \textcolor{dark-red}{\texttt{Max\_Torque\_Amplitude}} &
C &
50 ms  &
  \textbf{LDW Safety} Software Block  &
the lane assistance item is turned off 
\\\hline
\textcolor{harmonia-blue}{\texttt{Technical Safety Requirement 01-01-02}}  &
As soon as the LDW function deactivates the LDW feature, the 
  \textbf{'LDW Safety'}
  software block shall send a signal to the car display ECU to turn on a
  warning light&
C &
50 ms  &
  \textbf{LDW Safety} Software Block  &
the lane assistance item is turned off 
\\\hline
\textcolor{harmonia-blue}{\texttt{Technical Safety Requirement 01-01-03}}  &
As soon as a failure is detected by the LDW function, it shall deactivate the
  LDW feature and the 
  \textcolor{dark-red}{\texttt{LDW\_Torque\_Request}} shall be set to zero &
C &
50 ms  &
  \textbf{LDW Safety} Software Block  &
the lane assistance item is turned off 
\\\hline
\textcolor{harmonia-blue}{\texttt{Technical Safety Requirement 01-01-04}}  &
The validity and integrity of the data transmission for 
  \textcolor{dark-red}{\texttt{LDW\_Torque\_Request}}
  signal shall be ensured &
C &
50 ms  &
  \textbf{LDW Safety} Software Block  &
the lane assistance item is turned off 
\\\hline
\textcolor{harmonia-blue}{\texttt{Technical Safety Requirement 01-01-05}}  &
Memory test shall be conducted at start up of the EPS ECU to check for any
  faults in memory &
A &
ignition cycle  &
  \textbf{Memory Management Unit}   &
the lane assistance item is turned off 
\\\hline
\end{tabular}
\end{center}
\label{tab:tr0101}
\end{table}

% Instructions: 
% Fill in the technical safety requirements for the lane departure warning
% second functional safety requirement. We have provided the associated functional safety
% requirement in the table below. 
% 
% Hint: Most of the technical safety requirements will be the same. 
% At least one technical safety requirement will have to be slightly modified
% because we are talking about frequency instead of amplitude. These requirements were
% not given in the lessons

Functional Safety Requirement 01-02 with its associated system elements
(derived in the functional safety concept) is presented in Table~\ref{tab:func0102}.

\begin{table}[!htpb]
%\hspace*{-2.0cm}
\caption{Functional Safety Requirement}
\begin{center}
\scriptsize
\renewcommand{\arraystretch}{1.4}
\begin{tabular}{ L{1.8cm}|L{4.5cm}|L{2.2cm}|L{2.2cm}|L{2.2cm}  }
\hline
\rowcolor{black!10}
ID &
Functional Safety Requirement &
Electronic Power Steering ECU &
Camera ECU &
Car Display ECU
\\\hline
\textcolor{dark-green}{\texttt{Functional Safety Requirement 01-02}}  &
The lane keeping item shall ensure that the lane departure
oscillating torque frequency is below 
\textcolor{dark-red}{\texttt{Max\_Torque\_Frequency}}  &
\checkmark  &
  &
\\\hline
\end{tabular}
\end{center}
\label{tab:func0102}
\end{table}


Technical Safety Requirements related to Functional Safety Requirement 01-02
are presented in Table~\ref{tab:tr0102}.

\begin{table}[!htpb]
%\hspace*{-2.0cm}
\caption{Technical Safety Requirements for
  \textcolor{dark-green}{\texttt{Functional Safety Requirement 01-02}}}
\begin{center}
\scriptsize
\renewcommand{\arraystretch}{1.4}
\begin{tabular}{ L{1.8cm}|L{5.2cm}|L{0.7cm}|L{1.3cm}|L{1.8cm}|L{2.0cm}  }
\hline
\rowcolor{black!10}
ID &
Technical Safety Requirement  &
ASIL &
Fault Tolerant Time Interval  &  
Architecture Allocation  &  
Safe State \\\hline
\textcolor{harmonia-blue}{\texttt{Technical Safety Requirement 01-02-01}}  &
The LDW safety component shall ensure that the frequency of the 
  \textcolor{dark-red}{\texttt{LDW\_Torque\_Request}}
  sent to the 
  \textbf{'Final electronic power steering Torque'} component is below 
  \textcolor{dark-red}{\texttt{Max\_Torque\_Frequency}} &
C &
50 ms  &
  \textbf{LDW Safety} Software Block  &
the lane assistance item is turned off 
\\\hline
\textcolor{harmonia-blue}{\texttt{Technical Safety Requirement 01-02-02}}  &
As soon as the LDW function deactivates the LDW feature, the 
  \textbf{'LDW Safety'}
  software block shall send a signal to the car display ECU to turn on a
  warning light&
C &
50 ms  &
  \textbf{LDW Safety} Software Block  &
the lane assistance item is turned off 
\\\hline
\textcolor{harmonia-blue}{\texttt{Technical Safety Requirement 01-02-03}}  &
As soon as a failure is detected by the LDW function, it shall deactivate the
  LDW feature and the 
  \textcolor{dark-red}{\texttt{LDW\_Torque\_Request}} shall be set to zero &
C &
50 ms  &
  \textbf{LDW Safety} Software Block  &
the lane assistance item is turned off 
\\\hline
\textcolor{harmonia-blue}{\texttt{Technical Safety Requirement 01-02-04}}  &
The validity and integrity of the data transmission for 
  \textcolor{dark-red}{\texttt{LDW\_Torque\_Request}}
  signal shall be ensured &
C &
50 ms  &
  \textbf{LDW Safety} Software Block  &
the lane assistance item is turned off 
\\\hline
\textcolor{harmonia-blue}{\texttt{Technical Safety Requirement 01-02-05}}  &
Memory test shall be conducted at start up of the EPS ECU to check for any
  faults in memory &
A &
ignition cycle  &
  \textbf{Memory Management Unit}   &
the lane assistance item is turned off 
\\\hline
\end{tabular}
\end{center}
\label{tab:tr0102}
\end{table}


%\subsection{Lane Departure Warning (LDW) Verification and Validation Acceptance
%Criteria}

% OPTIONAL: For each technical safety requirement, identify both the verification and
% validation acceptance criteria. 
% "Validation" asks whether or not you chose the
% appropriate parameters. "Verification" involves testing to make sure the vehicle behaves
% as expected when the parameter value is crossed. 
% There is not necessarily one right answer. 
% Look at your verification and validation acceptance criteria from the functional
% safety concept for inspiration.


\subsection{Lane Keeping Assistance (LKA) Requirements}

% Instructions: Fill in the technical safety requirements for the lane keeping assistance
% functional safety requirement 02-01. We have provided the associated functional safety
% requirement in the table below. 
% 
% Hint: You can reuse the technical safety requirements from functional safety requirement 01-01. 
% But you need to change the language because we are now looking at a different system. 
% The ASIL and Fault Tolerant Time Interval are different as well.

Functional Safety Requirement 02-01 with its associated system elements
(derived in the functional safety concept) is presented in Table~\ref{tab:func0201}).


\begin{table}[!htpb]
%\hspace*{-2.0cm}
\caption{Functional Safety Requirement}
\begin{center}
\scriptsize
\renewcommand{\arraystretch}{1.4}
\begin{tabular}{ L{1.8cm}|L{4.5cm}|L{2.2cm}|L{2.2cm}|L{2.2cm}  }
\hline
\rowcolor{black!10}
ID &
Functional Safety Requirement &
Electronic Power Steering ECU &
Camera ECU &
Car Display ECU
\\\hline
\textcolor{dark-green}{\texttt{Functional Safety Requirement 02-01}}  &
The lane keeping item shall ensure that the lane keeping assistance torque is
applied for only 
\textcolor{dark-red}{\texttt{Max\_Duration}} &
\checkmark  &
  &
\\\hline
\end{tabular}
\end{center}
\label{tab:func0201}
\end{table}


Technical Safety Requirements related to Functional Safety Requirement 02-01 are
presented in Table~\ref{tab:tr0201}.


\begin{table}[!htpb]
%\hspace*{-2.0cm}
\caption{Technical Safety Requirements for
\textcolor{dark-green}{\texttt{Functional Safety Requirement 02-01}}}
\begin{center}
\scriptsize
\renewcommand{\arraystretch}{1.4}
\begin{tabular}{ L{1.8cm}|L{5.2cm}|L{0.7cm}|L{1.3cm}|L{1.8cm}|L{2.0cm}  }
\hline
\rowcolor{black!10}
ID &
Technical Safety Requirement  &
ASIL &
Fault Tolerant Time Interval  &  
Architecture Allocation  &  
Safe State \\\hline
\textcolor{harmonia-blue}{\texttt{Technical Safety Requirement 02-01-01}}  &
The LKA safety component shall ensure that the time interval of applying the  
  \textcolor{dark-red}{\texttt{LKA\_Torque\_Request}}
  is below 
  \textcolor{dark-red}{\texttt{Max\_Duration}} &
B &
500 ms  &
  \textbf{LKA Safety} Software Block  &
the lane assistance item is turned off 
\\\hline
\textcolor{harmonia-blue}{\texttt{Technical Safety Requirement 02-01-02}}  &
As soon as the LKA function deactivates the LKA feature, the 
  \textbf{'LKA Safety'}
  software block shall send a signal to the car display ECU to turn on a
  warning light&
B &
500 ms  &
  \textbf{LKA Safety} Software Block  &
the lane assistance item is turned off 
\\\hline
\textcolor{harmonia-blue}{\texttt{Technical Safety Requirement 02-01-03}}  &
As soon as a failure is detected by the LKA function, it shall deactivate the
  LKA feature and the 
  \textcolor{dark-red}{\texttt{LKA\_Torque\_Request}} shall be set to zero &
B &
500 ms  &
  \textbf{LKA Safety} Software Block  &
the lane assistance item is turned off 
\\\hline
\textcolor{harmonia-blue}{\texttt{Technical Safety Requirement 02-01-04}}  &
The validity and integrity of the data transmission for 
  \textcolor{dark-red}{\texttt{LKA\_Torque\_Request}}
  signal shall be ensured &
B &
500 ms  &
  \textbf{LKA Safety} Software Block  &
the lane assistance item is turned off 
\\\hline
\textcolor{harmonia-blue}{\texttt{Technical Safety Requirement 02-01-05}}  &
Memory test shall be conducted at start up of the EPS ECU to check for any
  faults in memory &
A &
ignition cycle  &
  \textbf{Memory Management Unit}   &
the lane assistance item is turned off 
\\\hline
\end{tabular}
\end{center}
\label{tab:tr0201}
\end{table}


%\subsection{Lane Keeping Assistance (LKA) Verification and Validation Acceptance Criteria}

% OPTIONAL: For each technical safety requirement, identify both the verification and
% validation acceptance criteria. 
% "Validation" asks whether or not you chose the
% appropriate parameters. 
% "Verification" involves testing to make sure the vehicle behaves
% as expected when the parameter value is crossed. There is not necessarily one right
% answer. Look at your verification and validation acceptance criteria from the functional
% safety concept for inspiration.


\section{Refinement of the System Architecture}

The refined architecture after specifying technical safety 
requirements is shown on Figure~\ref{fig:ref-arch-tech}

\begin{figure}[!htbp]
\includegraphics[width=1.00\linewidth]{graphic_asset_4}
\caption{Refined architecture}
\label{fig:ref-arch-tech}
\end{figure}

% Instructions: 
% Include the refined system architecture. Hint: The refined system architecture
% should include the system architecture from the end of the technical safety
% lesson, including all of the ASIL labels.

\section{Allocation of Technical Safety Requirements to Architecture Elements}

All technical safety requirements are allocated to the Electronic Power Steering ECU.

% Instructions: 
% We already included the allocation as part of the technical requirement tables.
% Here you can state that for this particular item, all technical safety
% requirements are allocated to the Electronic Power Steering ECU 

\section{Warning and Degradation Concept}

\begin{table}[!htpb]
%\hspace*{-2.0cm}
\caption{Warning and Degradation Concept}
\begin{center}
\scriptsize
\renewcommand{\arraystretch}{1.4}
\begin{tabular}{ L{1.8cm}|L{2.5cm}|L{4.2cm}|L{2.2cm}|L{2.2cm}  }
\hline
\rowcolor{black!10}
ID &
Degradation Mode &
Trigger for Degradation Mode &
Safe State invoked? &
Driver Warning
\\\hline
\textcolor{brown}{\texttt{WDC-01}}  &
Turn off the functionality &
Malfunction1, Malfunction2 &
Yes  &
Indication light on car display on\\\hline
\textcolor{brown}{\texttt{WDC-02}}  &
Turn off the functionality  &
Malfunction3 &
Yes  &
Indication light on car display on\\\hline
\end{tabular}
\end{center}
\label{tab:wdc}
\end{table}

% Instructions: 
% We've already identified that for any system malfunction, the lane assistance
% functions will be turned off and the driver will receive a warning light
% indication. The technical safety requirements have not changed how
% functionality will be degraded or what the warning will be.

% So in this case, the warning and degradation concept is the same for the technical safety
% requirements as for the functional safety requirements. You can copy the functional
% safety warning and degradation concept here.
 
% Oftentimes, a technical safety analysis will lead to a more detailed warning and
% degradation concept.
