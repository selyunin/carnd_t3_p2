%\chapter{Introduction}
%\label{ch:introduction}

\begin{table}[!htpb]
\caption{Hazard Analysis and Risk Assessment}
\begin{center}
\scriptsize
\renewcommand{\arraystretch}{1.4}
\hspace*{-2.0cm}
\begin{tabular}{ 
 |L{3.0cm}|
  L{3.0cm}|L{3.0cm}|L{3.0cm}|L{3.0cm}|L{3.0cm}|L{3.0cm}|L{3.0cm}|
  L{3.0cm}|L{3.0cm}|L{3.0cm}|L{3.0cm}|L{3.0cm}|L{3.0cm}|L{3.0cm}|
  L{3.0cm}|L{3.0cm}|L{3.0cm}|L{3.0cm}|L{3.0cm}|L{3.0cm}|
  L{3.0cm}|L{3.0cm}|}
\hline
\textbf{Hazard ID} & 
\multicolumn{7}{c|}{\textbf{Situational Analysis}} & 
\multicolumn{6}{c|}{\textbf{Hazard Identification}} &  
\multicolumn{6}{c|}{\textbf{Hazard Event Classification}} &
\multicolumn{2}{c|}{ \textbf{Determination of ASIL and Safety Goals}}
\\\hline
\rowcolor{black!10}
 &
% Situational Analysis
Operational Mode &
Operational Scenario  &
Environmental Details &
Situation Details &
Other Details (optional) &
Item Usage (function) & 
Situation Description &
% Hazard Identification
Function &
Deviation &
Deviation Details &
Hazardous Event (resulting effect) &
Event Details &
Hazardous Event Description &
% Hazard Event Classification
Exposure (of situation) &
Rationale (for exposition) &
Severity (of potential harm) &
Rationale (for severity) &
Controllability (of hazardous event) &
Rationale (for controllability) &
% Determination of ASIL and Safety Goals
ASIL Determination &
Safety Goal 
\\\hline
HA-001 &
% Situational Analysis
\textbf{OM3} Normal driving &
\textbf{OS04} Highway  &
\textbf{EN06} Rain (slippery road) &
\textbf{SD02} High speed &
 &
\textbf{IU01} Correctly used & 
High speed driving on a highway during rain (slippery road) &
% Hazard Identification
Lane Departure Warning (LDW) function shall apply an oscillating steering
torque to provide the driver with haptic feedback &
\textbf{DV04} Actor effect is too much. &
The LDW function applies an oscillating torque with very high torque (above limit). &
\textbf{EV00} Collision with other vehicle. &
High haptic feedback can affect driver's ability to steer as intended. The
driver could lose control of the vehicle and collide with another vehicle or
with road infrastructure. &
The LDW function applies too high an oscillating torque to the steering wheel (above limit). &
% Hazard Event Classification
\textbf{E3} medium probability &
Drivering in a highway may happen once a month or more &
\textbf{S3} Severe and life-threatening injuries &
High speed travelling on a highway &
\textbf{C3} Difficult to control or uncontrollable &
Overreaction of the steering wheel may surprise a driver
and cause over-reaction on steering which could lead to an
accident &
% Determination of ASIL and Safety Goals
\textbf{C} &
The oscillating steering torque from the lane departure warning function shall be limited
\\\hline
HA-002 &
% Situational Analysis
\textbf{OM3} Normal driving &
\textbf{OS03} Country Road &
\textbf{EN01} Normal conditions &
\textbf{SD02} High speed &
 &
\textbf{IU02} Incorrectly used & 
High speed driving driving on a country road under normal conditions, driver
is mis-using the system for autonomous driving &
% Hazard Identification
Lane Keeping Assistance (LKA) function shall apply the steering torque when
active in order to stay in ego lane &
\textbf{DV03} Function always activated &
LKA function is always activated &
\textbf{EV00} Collision with other vehicle &
Driver mis-uses the function for autonomous driving,
therefore unable to react quickly in critical situations &
The LKA function is always activated giving a driver
an illusion of autonomous driving. &
% Hazard Event Classification
\textbf{E2} low probability &
Driving on country road with high speed and mis-using the system &
\textbf{S3} Severe and life-threatening injuries &
High speed travelling on a country road &
\textbf{C3} Difficult to control or uncontrollable &
Driver could take hands off the steering wheel and
unable to react quickly when driving with high speed
 &
% Determination of ASIL and Safety Goals
\textbf{B} &
LKA function shall be time limited and the additional steering torque shall end
after a given timer interval so that the driver can not misuse the system for
autonomous driving.
\\\hline
HA-003 &
% Situational Analysis
\textbf{OM3} Normal driving &
\textbf{OS10} Road with construction site &
\textbf{EN01} Normal conditions &
\textbf{SD02} High speed &
 &
\textbf{IU01} Correctly used & 
High speed driving on a road with construction site, 
road signs and direction contradict lane lines&
% Hazard Identification
Lane Keeping Assistance (LKA) function shall apply the steering torque when
active in order to stay in ego lane &
\textbf{DV03} Function always activated &
LKA function is not deactivated when lane is not detected &
\textbf{EV-04} Front collision with obstacle &
Applied steering torque if lane is detected but 
no road signs contradict will confuse the driver, 
and make driving more difficult &
Applied torque to steer the car to the lane on a 
road which undergoes construction may lead to
collision with an infrastructure
 &
% Hazard Event Classification
\textbf{E3} medium probability &
Occurs once a month or more often for an average driver &
\textbf{S3} Severe and life-threatening injuries &
May lead to collision with infrastructure &
\textbf{C3} Difficult to control or uncontrollable &
Drivering may become more difficult to overcome
the functionality of the system &
% Determination of ASIL and Safety Goals
\textbf{C} &
LKA function shall be turned off if lane lines and 
road signs contradict each other
\\\hline
HA-004 &
% Situational Analysis
\textbf{OM3} Normal driving &
\textbf{OS04} Highway &
\textbf{EN07} Snow (slippery road) &
\textbf{SD02} High speed &
 &
\textbf{IU01} Correctly used & 
High speed driving on a winter-road (snow/ice)
&
% Hazard Identification
Lane Keeping Assistance (LKA) function shall apply the steering torque when
active in order to stay in ego lane &
\textbf{DV04} Actor effect is too much &
LKA function applies too much torque on a slippery road &
\textbf{EV-02} Side collision with other traffic &
Too much steering torque applied may lead to drift
and uncontrollable behavior &
Too much steering torque applied on a slippery road
may lead to drift on a high speed may lead to 
uncontrollable driving and collision
 &
% Hazard Event Classification
\textbf{E2} Low probability &
Occurs a few times a year for the great majority of drivers &
\textbf{S3} Severe and life-threatening injuries &
May lead to collision with other vehicles or infrastructure &
\textbf{C3} Difficult to control or uncontrollable &
Drivering may become more difficult go out of a drift
on a slippery road &
% Determination of ASIL and Safety Goals
\textbf{B} &
Steering torque from the LKA function shall be limited
depending on the environmental conditions and road
condition
\\\hline
%\textcolor{harmonia-blue}{\texttt{Software Safety Requirement 05-01}}  &
%A CRC verification check over the software code in the Flash memory shall be
%done every time the ignition is switched from off to on to check for any
%corruption of content.	
%&
%A &
%\texttt{MEMORYTEST}  &
%Activation\_status = 0
%\\\hline
%\textcolor{harmonia-blue}{\texttt{Software Safety Requirement 05-02}}  &
%Standard RAM tests to check the data bus, address bus and device integrity
%shall be done every time the ignition is switched from off to on (E.g.walking
%1s test, RAM pattern test. Refer RAM and processor vendor recommendations )
%&
%A &
%\texttt{MEMORYTEST}  &
%Activation\_status = 0
%\\\hline
%\textcolor{harmonia-blue}{\texttt{Software Safety Requirement 05-03}}  &
%The test result of the RAM or Flash memory shall be indicated to the 
%\textbf{LDW\_Safety}
%component via the 
%\textcolor{dark-red}{\texttt{test\_status}} signal	
%&
%A &
%\texttt{MEMORYTEST} &
%Activation\_status = 0
%\\\hline
\end{tabular}
\end{center}
\label{tab:sr05}
\end{table}
